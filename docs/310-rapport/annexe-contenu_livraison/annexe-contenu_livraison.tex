\chapter{Contenu de la livraison}

\section{Arborescence de l'archive}

La livraison contient les éléments suivants :
\dirtree{%
.1 (dossier racine).
	.2 docs/\DTcomment{Documents de la livraison}.
	.2 run/\DTcomment{Exécutable compilé et son environnement d'exécution}.
		.3 BAGOmaze\DTcomment{Exécutable}.
		.3 rsrc/\DTcomment{Dossier contenant les grilles statiques}.
	.2 sources/\DTcomment{Sources du logiciel}.
}

\section{Compilation des sources}

Le projet utilise \verb|qmake| et plusieurs autres dépendances.

Par commodité, un script \verb|install-dependecies-dev.sh| est disponible dans le dossier \verb|sources/BAGOmaze/| et permet d'installer les dépendances nécessaires à la majorité des systèmes Linux basés sur Debian comme Ubuntu.

Pour compiler les sources :
\begin{enumerate}
	\item Se rendre dans le dossier \verb|sources/BAGOmaze| ;
	\item Créer un sous-dossier \verb|build| ;
	\item Ouvrir un terminal dans le dossier \verb|build| ;
	\item Exécuter la commande \verb|qmake ..| ;
	\item Exécuter la commande \verb|make|
\end{enumerate}

L'exécutable est automatiquement déplacé dans le dossier \verb|BAGOmaze/out/app/|.
